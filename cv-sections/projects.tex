%----------------------------------------------------------------------------------------
%	SECTION TITLE
%----------------------------------------------------------------------------------------

\cvsection{Proyectos destacados}

%----------------------------------------------------------------------------------------
%	SECTION CONTENT
%----------------------------------------------------------------------------------------

\begin{cventries}

%------------------------------------------------

%\cventry
%{Creador} % Affiliation/role
%{\href{https://github.com/AythaE/twitter_search}{Twitter\_search}} % Organization/group
%{} % Location
%{} % Date(s)
%{ % Description(s) of experience/contributions/knowledge
%\begin{cvitems}
%\item {Programa en Python para recuperar tuits buscando alguna palabra o hashtag y exportarlos a un archivo .csv para realizar un análisis de redes sociales con estos. Usa el módulo Tweepy para conectarse a la API de Twitter y exporta los tuits en un formato que puede ser directamente interpretado por la herramienta Gephi.}
%\end{cvitems}
%}

%------------------------------------------------


\cventry
{Co-Creador} % Affiliation/role
{\href{https://github.com/AythaE/Estudiantes-TID}{Estudiantes-TID}} % Organization/group
{} % Location
{} % Date(s)
{ % Description(s) of experience/contributions/knowledge
	\begin{cvitems}
		\item {Proyecto de análisis inteligente de datos que partiendo de un dataset con datos demográficos, sociales, personales y emocionales de estudiantes portugueses de instituto entre los 15 y los 22 años pretende caracterizar los perfiles del consumidor y el no consumidor habitual de alcohol. Para ello emplea el entorno Knime y algunos scripts en R desarrollando todo el proceso típico de un proyecto de análisis de datos: partiendo de un análisis exploratorio, planteando diversas técnicas como la clasificación o las reglas de asociación y extrayendo conclusiones de estas.}
	\end{cvitems}
}

%------------------------------------------------

\cventry
{Creador} % Affiliation/role
{\href{https://github.com/AythaE/CC-servicios_y_aplicaciones/tree/master/Practica2_PaaS}{Restaurantes-Docker}} % Organization/group
{} % Location
{} % Date(s)
{ % Description(s) of experience/contributions/knowledge
	\begin{cvitems}
		\item {Despliegue de una aplicación web usando Nginx, Django y MongoDB en contenedores Docker. Partiendo de una colección de restaurantes permite el listado, la búsqueda y la incorporación de nuevos restaurantes a la base de datos. }
		%El hecho de estar preparada para funcionar en Docker permite su rápido despliegue y aporta diversas ventajas como independencia funcional, posibilidad de replicar contenedores y balancear la carga entre estos, etc...}
	\end{cvitems}
}


\end{cventries}